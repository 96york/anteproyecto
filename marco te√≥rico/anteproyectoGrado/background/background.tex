\chapter{Justificación}

\label{ch:background}

\section{Desarrollo de aplicación móvil para sustentar una idea de negocio}

Está demostrado que hoy en día la tendencia de concentración de usuarios de software se da en el ámbito móvil. Se ha visto que muchas compañías son capaces de monetizar una idea a partir del flujo de un número amplio de usuarios. Por ejemplo, a través de publicidad. Hay entidades dispuestas a pagar por publicidad, por ejemplo Universidades. Sobretodo es importante poder segmentar el mercado objetivo a quién dirigir la publicidad.

Sabiendo esto, es una gran idea sustentar una idea de negocio a través de los canales más frecuentados, en el caso de software es idóneo apuntar al ámbito móvil. Pues al buscar obtener un número elevado de usuarios es posible monetizar no sólo a través del negocio en si mismo, sino en áreas como el manejo de publicidad. Según un estudio Realizado por Asomovil en el 2014 los usuarios de Smartphones reportados superan los 14.4 millones en Colombia. De acuerdo al estudio, también es posible observar una tendencia a incrementar estas cifras pues del año 2013 al 2014 se registró un aumento del   5% Básicamente es una gran estrategia. Debido al gran flujo de usuarios con distintos segmentos de interés.[1]

Viendo esta oportunidad de negocio, hay que apuntar a un mercado de alta convergencia, como lo es el turismo. Al aplicar este modelo de negocio a un campo de acción tan grande, buscamos obtener un número alto de usuarios. Se pretende vender publicidad a agencias de turismo, hoteles y agencias de transporte. Según cifas del ministerio de comercio, industria y turismo. En el 2015 ingresaron a Colombia 4.447.004 viajeros que son un gran segmento de mercado objetivo.[2]


Referencias
